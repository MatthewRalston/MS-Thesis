% This is the Chapter 2 - Methods file (chap2.tex)

\chapter{Methods}


\section{Culture}
Wild type \textit{Clostridium acetobutylicum} ATCC 824 was cultured anaerobically in 4L New Brunswick Scientific BioFlo 310 bioreactors at \SI{37}{\degreeCelsius}, pH >= 5.0, \SI{200}{\milli\liter\per\minute} N\textsubscript{2} and 200rpm agitation in a defined \textit{Clostridia} growth medium, as described previously \cite{39}. When the cultures were grown to A\textsubscript{600}=1, the N\textsubscript{2} flow rate was decreased to \SI{50}{\milli\liter\per\minute} and cultures were either stressed to a final concentration of \SI{60}{\milli\Molar} \textit{n}-butanol, \SI{40}{\milli\Molar} potassium butyrate, or left unstressed. \SI{15}{\milli\liter} samples were acquired at 15, 75, 150, and 270 minutes after treatment and OD synchronization. Samples were centrifuged at 8,000rpm, \SI{4}{\degreeCelsius} for 20 minutes. After discarding the supernatant, cell pellets were then immediately frozen at \SI{-85}{\degreeCelsius}.

\section{RNA preparation}
\label{methods:RNA_prep}
RNA was extracted by first washing the cell pellets in 1mL of RNase-free SET buffer (25\% sucrose, \SI{50}{\milli\Molar} EDTA [pH 8.0], \SI{50}{\milli\Molar} Tris-HCl [pH 8.0]) before resuspending cells in a \SI{220}{\milli\liter} solution of RNase-free SET buffer containing \SI{4.55}{\Units\per\milli\liter} proteinase K and \SI{20}{\milli\gram\per\milli\liter} lysozyme and incubating for 6 minutes. Resuspended cells were vortexed with 40mg of RNase-free glass beads ($\leq$\SI{106}{\micro\metre}) at maximum speed and room temperature for 4 minutes. Each sample was mixed immediately with \SI{1}{\milli\liter} of ice-cold QIAzol (Qiagen, Valencia, CA, USA) and then \SI{200}{\micro\liter} of ice-cold chloroform, mixing well with each addition. After a 3 minute room temperature incubation, samples were centrifuged at 11,000rpm and \SI{4}{\degreeCelsius} for 15 minutes. The aqueous phase was then mixed with \SI{1.3}{\milli\liter} of ice-cold ethanol before transferring to a miRNeasy Mini spin-column (Qiagen, Valencia, CA, USA) and centrifuging at 11,000rpm and \SI{4}{\degreeCelsius} for 15 seconds.

Next, \SI{700}{\micro\liter} of RWT buffer was added to the column, before centrifuging at 11,000rpm and \SI{4}{\degreeCelsius} for 15 seconds, discarding the collection tube and transferring the column to a fresh collection tube. The column was washed twice with \SI{500}{\micro\liter} of RPE buffer before centrifuging at 11,000rpm and \SI{4}{degreeCelsius} for 15 seconds each. The membrane was then dried with an additional centrifugation step at 11,000rpm and \SI{4}{degreeCelsius} for 1 minute. The RNA was eluted twice by incubating with \SI{50}{\micro\liter} of nuclease-free water for 1 minute and eluting for 1 minute at 11,000rpm and \SI{4}{degreeCelsius}.

After quantification on a Nanodrop ND-1000, samples were then precipitated in 0.3M sodium acetate and 75\% ethanol overnight, centrifuged at 14,000 rpm for 30 minutes, washed twice with \SI{400}{\micro\liter} ice-cold 70\% ethanol, and rehydrated in \SI{50}{\micro\liter} RNase-free water. Next, samples were treated with the Turbo DNA-free kit (Ambion, Austin, TX, USA). \SI{5}{\micro\liter} of 10X Turbo DNase buffer and \SI{1}{\micro\liter} of Turbo DNase (2U\SI{}{\per\micro\liter}) were added to each sample before incubating at \SI{37}{degreeCelsius} for 30 minutes. Next, \SI{5}{\micro\liter} of DNase inactivation reagent were added to each sample, mixing occasionally for 5 minutes. The samples were then centrifuged at 10,000rpm and \SI{4}{degreeCelsius} for 90 seconds, precipitating the DNase. The samples were moved to fresh \SI{1.5}{\micro\liter} tubes.

Samples were then precipitated, washed twice more with 70\% ethanol, and resuspended in \SI{20}{\micro\liter} of nuclease-free water, requantified, and aliquoted for quality analysis with the BioAnalyzer platform (Agilent, Wilmington, DE, USA), and \SI{10}{\micro\gram} aliquots in \SI{10}{\micro\liter} samples were stored at \SI{-85}{\degreeCelsius}.

\section{RNA enrichment, RNA-seq library preparation, and Sequencing}
Ribosomal RNA was removed with the MicrobExpress kit (Ambion, Austin, TX, USA) according to their protocol. Briefly, beads were prepared by taking \SI{50}{\micro\liter} for each sample, washing with an equal volume (\SI{50}{\micro\liter}) of water capturing for 5 minutes on a MagnaSphere (Promega, Madison, WI, USA) magnetic stand and aspirating. Subsequently, the beads were resuspended in an equal volume (\SI{50}{\micro\liter} each) of binding buffer and capturing as above. The beads were then resuspended in an equal volume (\SI{50}{\micro\liter} each) of binding buffer and warmed to \SI{37}{\degreeCelsius}. Next, \SI{200}{\micro\liter} of binding buffer was added to each \SI{10}{\micro\gram} RNA aliquot with \SI{4}{\micro\liter} of capture oligo mix. The mixture was warmed to \SI{70}{\degreeCelsius} for 10 minutes, then cooled to \SI{37}{\degreeCelsius} for 15 minutes. Next, the rRNA was captured by mixing \SI{50}{\micro\liter} of beads with each sample, incubating for 15 minutes at \SI{37}{\degreeCelsius}, and capturing as above. The enriched RNA was transferred to a fresh \SI{1.5}{\milli\liter} tube. The beads were then washed with \SI{100}{\micro\liter} of pre-warmed (\SI{37}{\degreeCelsius}) wash solution, incubating on the magnetic stand for 5 minutes, and adding the wash solution to the enriched RNA. The samples were then ethanol precipitated at \SI{20}{\degreeCelsius} overnight with \SI{35}{\micro\liter} of \SI{3}{\Molar} Sodium Acetate, \SI{5}{\milli\gram\per\milli\liter} Glycogen, and \SI{1175}{\micro\liter} of chilled 100\% ethanol. The samples were washed twice with 70\% ethanol and resuspended in \SI{25}{\micro\liter}. The samples were enriched further by repeating the MicrobExpress treatment. Small 10-\SI{100}{\nano\gram} aliquots were analyzed at each step with the BioAnalyzer to monitor enrichment.

Selected samples were enriched further with Terminator 5'-phosphate dependent exonuclease kit (Epicentre, Madison, WI, USA). Terminator Exonuclease \SI{1}{\micro\liter} (1U\SI{}{\per\micro\liter}) was added with \SI{2}{\micro\liter} 10X Buffer A to each RNA sample. The reaction was run in a      thermocycler for 60 minutes at \SI{30}{\degreeCelsius}. The reaction was terminated with the addition of \SI{1}{\micro\liter} of \SI{100}{\milli\Molar} EDTA and ____ Tris HCl at pH 8.0. The samples were then purified by ethanol precipitation (\SI{0.3}{\Molar} Sodium Acetate and 75\% ethanol) with two 70\% ethanol washes, as above.
Enriched RNA was quantified as above and assessed for quality with the BioAnalyzer platform (Agilent, Wilmington, DE, USA). High quality samples were used to prepare RNA-seq libraries with the ScriptSeq v2 library preparation kit and indexed PCR primers (Epicentre, Madison, WI, USA). Briefly,  \SI{1}{\micro\liter} of fragmentation solution and \SI{2}{\micro\liter} of cDNA synthesis primer was added to \SI{50}{\nano\gram} of RNA and the solution was fragmented for 5 minutes at \SI{85}{\degreeCelsius} in a _____ thermocycler. To each reaction, \SI{0.5}{\milli\Molar} of Dithiothreitol, \SI{3}{\micro\liter} of cDNA synthesis premix, \SI{0.5}{\micro\liter} StarScript Reverse Transcriptase. is added to each sample and run with the following cycle: 5 minutes at \SI{25}{\degreeCelsius}, 20 minutes at \SI{42}{\degreeCelsius}. After cooling each reaction to \SI{37}{\degreeCelsius}, \SI{1}{\micro\liter} of finishing solution was added, incubating for 10 minutes. The RNA is degraded by fragmenting further for 3 minutes at \SI{95}{\degreeCelsius}, cooling to \SI{25}{\degreeCelsius}. The first strand cDNA is di-tagged by adding \SI{7.5}{\micro\liter} of terminal tagging premix and \SI{0.5}{\micro\liter} of DNA polymerase. The terminal tagging reaction is run at \SI{25}{\degreeCelsius} for 15 minutes and \SI{95}{\degreeCelsius} for 3 minutes. The di-tagget cDNA is then purified with the AMPure XP bead system (Beckmann Coulter, Brea, CA, USA). First, the library is mixed with \SI{45}{\micro\liter} of homogenous bead mixture. After thorough mixing, each solution is transferred to a \SI{1.5}{\milli\liter} tube and the library is captured with the magnetic stand and the supernatant aspirated. Each library is then washed twice with \SI{200}{\micro\liter} of 80\% ethanol. After resuspending in \SI{24.5}{\micro\liter} of nuclease-free water, the beads are captured and each library is transferred to a new \SI{200}{\micro\liter} microfuge tube. Adapters are added to the di-tagged cDNA during PCR by adding \SI{25}{\micro\liter} FailSafe Premix E, \SI{1}{\micro\liter} forward primer, \SI{1}{\micro\liter} of ScriptSeq v2 indexed reverse PCR primer, \SI{0.5}{\micro\liter} of FailSafe Polymerase. The PCR conditions are as follows: cycles of 30 seconds of \SI{95}{\degreeCelsius}, 30 seconds of \SI{55}{\degreeCelsius}, and 3 minutes of \SI{68}{\degreeCelsius}. After 12 cycles, the reaction terminates with a 7 minute incubation at \SI{68}{\degreeCelsius} before purifying the library with the AMPure system, as above.  Libraries were multiplexed and sequenced for 76 cycles over two lanes of an Illumina HiSeq 2500 at the University of Delaware Sequencing and Genotyping Center (Newark, DE, USA).
\section{Data Processing, Alignment}
Paired-end sequencing resulted in 749,709,771 pairs of 76 bp reads which are deposited in the Sequence Read Archive ( _____ ). Summary statistics for the libraries are shown in table/appendix ( ___ ). The basic bioinformatic processing pipeline is described on \href{https://github.com/MatthewRalston/NGS_scripts}{Github}. In brief, the fastq headers are briefly pre-processed for downstream applications. Then, remaining sequencing adaptors were removed from the reads with Trimmomatic\cite{47}. Base quality is adjusted by trimming to the minimum Phred base quality of 20, corresponding to a base-calling error probability of 0.01. Before aligning to the \textit{Clostridium acetobutylicum} ATCC 824 genome, the data were subjected to \textit{in silico} ribosomal RNA removal by aligning the reads to the rRNA sequences with Bowtie 2.1.0\cite{17}. The unmapped reads were then aligned to the genome and megaplasmid sequences (NC_003030.1 and NC_001988.2). The alignment files were then cleaned, sorted, indexed, and validated before removing duplicate reads with SAMtools\cite{19} and Picard\cite{40}.

\section{Coverage analysis}
Coverage vectors for each strand were calculated with BEDtools\cite{49}. Coverage vectors for each transcript were then acquired with a custom Ruby script. Summarization and visualization of these data was performed in R\cite{59}, circos, rails, and d3.

\section{Transcriptome Assembly, Quality, and Annotation}
Reference and \textit{de novo} assembly was done with Trinity\cite{58}. Fastq files were modified by appending the second column of the fastq Casava 1.8+ header to the first column before processing and alignment. Next, the resulting alignment files were merged and sorted before appending the pair information (/1 or /2) according to the Trinity documentation. 
To assess the assemblies, I have contributed to a transcriptome assembly assessment software project: \href{https://github.com/MatthewRalston/transrate}{Transrate}. This software assesses transcriptome assemblies by calculating general assembly statistics, coverage statistics, and agreement with the reference proteome. I have made several additions to this software. Specifically, unpaired reads and strand specific alignment were integrated into the coverage/alignment statistics. Additionally, singleton reads were produced from the alignment process and were then further assessed for possible sources of contamination. Finally, the assembly itself was aligned to the reference genome, assuring the validity of the assembly and the identity of the assembled transcripts. The assembly, in fasta format, was then aligned to the genome with blat, converted to bed format, processed, converted to genePred and ultimately to gtf format. The gtf format assembly was then combined with the reference proteome for comparison.
Assembly statistics were produced with a ruby script, grouping transcripts as 'standard' (reference-ORF containing) or 'novel.' For the standard transcripts, UTR lengths and IGRs were calculated and compared with both the reference annotation and according to the operon organization by Paredes et al (CITATION).
Assembly quality was assessed with specific examples of canonical genes and curated through a customized genome browser. These regions were probed for agreement between known transcriptional start sites, transcript sizes, ORF boundaries, promoter, and terminator annotations.
ORFs were predicted with transdecoder and subsequently annotated in RAST.


\section{Digital Gene Expression, Principal Components Analysis, and Differential Expression}
Read counts per transcript were quantified with HTSeq\cite{48}. Raw count data were visualized and normalized in R. The data were regularized following the conservative approach of DESeq2\cite{52}, \cite{53}. The processed data was subject to Principal Components Analysis using the rgl library in R, and results were added to an interactive webpage. A Wald test was used to test for differential expression. Calculations and visualizations were done in R with various packages(OTHERS)\cite{55}. Data were also processed manually for visualization in Circos graphs\cite{50}.

\section{Gene Expression Clustering and Visualization}
Regularized data were normalized or converted into Kendall, Pearson, and Spearman correlation matrices in R. The data were used as input to a parameter sweep with my href{https://github.com/MatthewRalston/OPTICS-Automatic-Clustering}{implementation} of the OPTICS clustering algorithm. I have tweaked the source code of the automatic feature extraction to be closer to the original algorithm (REFERENCE). Additionally, I added per-cluster metrics, described in the project README. This allowed me to visualize the results of the parameter sweeps and select the data input and parameter values that produce the best clusters. Exploratory data analysis was done with 'knitr' for HTML report generation.

\section{Web Content}
Interactive web material was generated using a mixture of Ruby on Rails, javascript, HTML, and CSS. The d3 library was used for dynamic content and interactivity. The genome browser was designed as a Ruby on Rails application with a PostgreSQL backend. After query optimization, the UI was designed with biologists in mind, to enable exploration of annotations and coverage data. Additionally, users are able to upload new annotations to the database as they are produced. Circos was used to generate the larger circular plot visualization. These web pages are hosted on github for access by collaborators.


