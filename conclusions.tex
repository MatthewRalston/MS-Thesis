% This is the Chapter 5 - Conclusions file (conclusions.tex)

\chapter{Conclusions}

Renewables research revolves around the development of a feedstock-flexible chassis organism that requires minimal engineering for biofuels production, such as \textit{C. acetobutylicum}. The development of this organism requires a complete genome annotation consisting of ORFs, promoters, terminators, and transcript boundaries. The existing annotation of this microbe is largely the result of ORF predictions from antiquated gene models and is consequently incomplete. Genomic and molecular research will be more efficacious with an accurate genome annotation.

High-throughput transcriptomic methods such as RNA-seq are ideal to update this annotation, despite the well documented challenges related to this platform. Many of these challenges have not been addressed by the literature, leading to poor data utilization rates(Table \ref{table:study_compare}). The absence of standards for bacterial transcriptome mapping studies provided the opportunity to develop an innovative technique to explicitly address false positive and false negative signal in sequencing datasets.

Several problematic issues, including rRNA and RNA degradation, were addressed by developing a laboratory workflow and quality control system prior to deep sequencing. The dataset was cleaned for errors, biases, and contaminants for proper quantification of the sensitivity. 450M properly-paired reads provided \textgreater 9000 fold-coverage of the \textit{C. acetobutylicum}, with a median per-base sequencing depth of 156x. This method even detected low-level background signals, an issue affecting deep sequencing studies that leads to false positive errors, yet ignored by studies in the microbial community. 

To identify and treat these issues, a fast and flexible genome browser was constructed. The genome browser visualized the background signals and misassemblies that were detectable in assembly statistics. Genome-wide promoter predictions revealed the prevalence of $\sigma_{A}$-promoter elements in the AT-rich \textit{C. acetobutylicum} genome, a potential source for the background signals. An integrative analysis method was developed to correct the misassemblies where necessary by including sequencing depth, complexity, Rho-independent terminators, and promoter motifs in the annotation visualization and curation method.

Most examples required little to no curation and showed excellent precision and accuracy with respect to previous studies. Even challenging edge cases involving multiple transcription start sites had excellent signal to noise ratio and consequently simple corrections. A proof-of-principle curation of the pSOL1 megaplasmid produced ideal assembly statistics, including a median transcript size of 1.4kb consistent with the reported average transcript lengths in \textit{E. coli}. A total of 86 reference-ORF containing transcripts and 24 novel transcripts were identified.

By explicitly addressing several issues related to false positive and false negative signals, a sequencing protocol and integrative analysis method was developed. This method lead to the first strand-specific transcriptome assembly in the genus \textit{Clostridia}. The technique described by this work is applicable in any bacterial species where a genome sequence is available. While the unprecedented sequencing depth of this study lead to false positive results in the initial assembly, the integrative curation method provided both precision and accuracy in transcript boundary determination. 

\chapter{Future Work}
This study provided \textit{C. acetobutylicum} transcript boundaries for future molecular and genomics studies. The method used in the proof of principle curation of the pSOL1 megaplasmid should be extended to the entire \textit{C. acetobutylicum} chromosome. It is reasonable to expect that the transcript and UTR sizes for the whole genome should be similarly improved. The transcript boundaries could improve expression estimates and differential expression analyses. Beyond comparing with previous microarray studies, novel transcripts discovered here could be discovered to be stress responsive. Such findings would be natural targets for future targeted or whole genome stress response analyses. 

The transcript boundaries also facilitate regulatory motif identification. Besides the promoter motifs and transcription factor binding sites described here, new motifs could exist upstream of transcription start sites of clusters of co-regulated genes. Differential promoter usage can be investigated using the genome browser and gene-specific techniques. Gene networks in \textit{C. acetobutylicum} inevitably possess transcription activation systems and will be ready to be explored with a complete genome annotation.

In fact, differential expression and motif analyses can be combined in an interesting way. By coupling differentially expressed genes, their expression profiles, and clustering algorithms, patterns of co-expression and perhaps co-regulation may be identified. The performance of such an approach naturally depends on normalization approaches (to make profiles reasonably comparable) and clustering theory. While some approaches have been suggested for expression profiles specifically, the performance of any approach is ultimately determined by these two factors and thus all options should be explored. 

Additional research can be done with comparative genomics, including re-annotation of the genome and transcriptome for protein coding ORFs. With the example of the missing CAC2079 gene for example, there may be substantially transcribed and protein coding regions that require comparative analysis to identify metabolite exporters, two-component systems, and more. In fact, the RAST annotation system\cite{160} can be used to annotate transcriptomes, with some clever scripting and knowledge of its features.

Finally, additional analyses should be done for RNA hybridization and small RNA target prediction. Many recently identified stress responsive small RNAs\cite{39} have unknown targets and functions. Exploration of their roles has been prohibited by undefined UTR structures and thus the thermodynamics of interaction with their partners. A complete genome annotation provides these information and an additional tractable problem for \textit{C. acetobutylicum} researchers to explore.

The first assembly and annotation provided by this work presents a number of opportunities for additional work in \textit{C. acetobutylicum}. Increasing temperatures, CO$_{2}$ levels, and energy prices provide ethical and economical incentives to explore renewables research in this organism. A small number of intrinsic conditions have explored in this study, including the exponential and transition stages of growth, and stress responses to metabolite stress. Additional conditions may display alternate gene sets, expanding the complexity of the transcriptome beyond what is described here. Novel transcripts identified in the megaplasmid and chromosome require annotation and molecular study. If enzymes are discovered, metabolic models should be updated. Differential expression experiments are an increasingly useful method for understanding transcriptomic dynamics. Such studies in \textit{C. acetobutylicum} directly benefit from a wider and more complete genome annotation. This work facilitates future research in \textit{C. acetobutylicum} through the genome browser, which can incorporate future annotations and display expression data in a fast, flexible, and data-dense manner.



