% This is the file for the background chapter of my thesis

\chapter{Background}

\section{Renewable Chemicals: A 21\textsuperscript{st} Century Challege}
Climate change research\cite{95} and petrochemical exploration\cite{120,121} suggest that escalated weather variation, sea levels, and atmospheric and oceanic temperatures will accompany steep increases in the cost of petroleum. Renewable chemical platforms will be an increasingly economical solution to climate change. Renewable biofuels can be carbon neutral sources of energy. Renewable biochemical processes can actually behave as carbon sinks, with net accumulation of CO$_{2}$ in chemicals and biomass after subtracting processing energy requirements.

Many renewable biochemical systems revolve around a biocatalyst, a microbe that can convert low cost substrates into fuels or chemicals.\cite{3,4,13,16,24,26} This production system, frequently referred to as a ``biorefinery,'' requires a microorganism with a wide range of potential feedstocks and a natural biofuel producing metabolism. Biofuels with comparable energy density and hygroscopicity to conventional fuels are desirable for infrastructure compatibility. The butanol-producing bacteria \textit{Clostridium acetobutylicum} consumes a number of sugars and hydrolysates\cite{101} and produces butanol, a direct gasoline replacement.\cite{122}

\textit{C. acetobutylicum} is a historically industrial solvent producer.\cite{123} It consumes hemicellulose, a variety of simple and complex carbohydrates, and hydrolysates.\cite{101,123} Also known as the Weizmann organism, \textit{C. acetobutylicum} converts these substrates into solvents through an acetone, butanol, and ethanol (ABE) fermentation.\cite{101,123} Importantly, these cells synthesize most amino acids with ammonium salts as a nitrogen source, requiring only a minimal defined medium.\cite{124,39} This microbe meets the requirements for low-cost non-food feedstocks and infrastructure compatibility. Therefore, \textit{C. acetobutylicum} is an excellent chassis organism for an integrated biorefinery and is the system of study in this work.

\textit{C. acetobutylicum} has a number of intrinsic advantages that minimize the engineering efforts required for bioprocess development. It is one of over 17,000 bacteria with sequenced genomes.\cite{89,90,91} For example, current metabolic models\cite{100} are used for sophisticated metabolic analyses, such as $C^{13}$ metabolic flux analysis.\cite{99} A model for the solventogenic \textit{Clostridia}, \textit{C. acetobutylicum} is a reasonably well studied organism with industrial potential.

Prior to the genomic era, targeted studies in this organism revealed the specific loci for solvent formation,\cite{62,63,64,66,68,72,73} sporulation,\cite{125} and canonical heat-shock operons.\cite{76,80} These genes were typically cloned, sequenced, and investigated with gene-specific transcriptomic techniques. However, only the mechanisms of the unique metabolic systems (e.g. solvent formation) have been investigated in detail; many proteomic and transcriptomic mechanisms behind the \textit{C. acetobutylicum} regulatory networks remain unknown.\cite{126,42} The mechanisms for the majority of the genetic and metabolic systems in \textit{C. acetobutylicum} are largely inferred from homology, often an appropriate assumption.

That being said, many interesting characteristics of \textit{C. acetobutylicum} are unique to solventogenic \textit{Clostridia}. Of particular interest for renewable fuel research is its solvent stress-response, which may be uniquely adapted to its solvent-producing metabolism. A number of stress-response systems exist for specific stresses while broader systems can respond to multiple stressors. By exploring the knowledge of these systems in \textit{C. acetobutylicum}, gaps in understanding can be identified for this work to explore. In the next section, stress-response systems are reviewed, demonstrating opportunities for the discovery of novel transcriptomic features.

\section{Biofuel/Solvent Tolerance and the Bacterial Stress Response}


Bacteria respond to a wide variety of intrinsic and extrinsic challenges with stress response systems.\cite{11,24,77} For example, nutrient deprivation, osmotic shock, temperature fluctuation, and high chemical concentrations are common in their natural environments.\cite{11} Cells activate specific or general response networks after these insults. \textit{C. acetobutylicum} is a naturally solventogenic bacterial species and could possess biofuel or solvent-tolerance genes that would likely be regulated by general or specific stress response systems. Such genes would be natural targets for biosystems engineering, improving the tolerance of engineered strains to high biofuel titers. 

Unfortunately, knowledge of these systems is incomplete in \textit{C. acetobutylicum} and no unique solvent tolerance mechanisms, such as solvent exporters, have been identified. The stress response is an important system for biotechnology, but the annotations of \textit{Clostridia} genomes require improvement for understanding of these and other systems. Here we review these systems and their mechanisms to understand how a solvent stress response system would function.


\subsection{Specific Stress Response Systems}
Specific stress-responses are designed to mitigate the negative effects of a particular stressor. These systems typically contain a detection mechanism for a molecular stressor, such as antibiotic compounds,\cite{127} or its effects, such as DNA damage.\cite{128}

Antibiotic resistance is an example of a specific response system that detects a molecular stressor. In the presence of organic compounds of the $\beta$-lactam\cite{127} or tetracycline families,\cite{130} bacteria activate antibiotic resistance genes that either export or modify the organic compounds to prevent their action. Some of these antibiotic-resistance genes are part of operons that possess antibiotic-detecting repressors.\cite{127} Upon detection of the antibiotic agent, a conformation change triggers derepression of the antibiotic resistance gene, resulting in an antibiotic resistant phenotype. This stress response system helps bacteria to quickly and specifically respond to a family of antibiotic compounds. Interestingly, multidrug resistance genes can contribute to solvent tolerance in \textit{P. aeruginosa}\cite{202} and \textit{E. coli}.\textit{203}

A second example of a specific stress response is from \textit{D. radiodurans}, which has an unparalleled resistance to ionizing radiation.\cite{129} In response to breaks in DNA, \textit{D. radiodurans} repairs the multiple copies of its genome, enabling growth, viability, and survival at over 5,000 Gy of radiation.\cite{129} This system responds instead to DNA damage, a symptom of $\gamma$-radiation. The extreme tolerance of \textit{D. radiodurans} to radiation is a byproduct of a specialized system for DNA repair,\cite{129} augmented from the standard DNA repair system.\cite{128} This system contains a detection system, non-homologous recombination, and a specialized DNA repair system allows \textit{D. radiodurans} cells to survive and repair hundreds of insults to its genome. 

If specific stress response systems exist for solvent stress (e.g. solvent-exporting efflux pumps) in \textit{C. acetobutylicum}, they would be desirable targets for biosystems engineering. Solvent tolerance pumps, such as the SrpABC efflux pump of \textit{P. putida}, could enable a butanol-tolerant phenotype.\cite{201} A solvent efflux system was not described in the original genome annotation,\cite{91} although gene models have changed significantly in the previous 14 years. An updated, improved genome annotation could enable the identification of solvent efflux genes.

In addition, stress responsive small RNAs (sRNAs) have been described,\cite{39} although their roles, regulation, and conservation remain unknown. This discovery suggests that there are previously unknown active regions of the \textit{C. acetobutylicum} genome that require investigation. Naturally, differential expression experiments are desirable to understand the dynamics of sRNAs, their targets, and other novel transcripts. However, the statistical treatment of measurements for such experiments is complex and requires reliable estimates of transcript expression.\cite{111,112,197} While these estimates would be impossible to acquire with the original bioinformatically-predicted genome annotation, an improved genome annotation could benefit these investigations in several ways. First, transcriptome mapping could identify novel stress-responsive transcripts that could be co-regulated with the previously discovered small RNAs.\cite{39} Novel stress-responsive transcripts could encode ORFs homologous to exporters or efflux pumps. Precise transcript boundaries could improve gene-expression estimates by counting reads at the transcript level, as opposed to the ORF level. Therefore, an improved annotation could reveal novel components of the stress response and would support differential expression investigations of these systems.

In addition, an improved genome annotation complete with transcription start sites and regulatory regions would facilitate research on stress-responsive genes and associated regulatory motifs. Regulatory motifs could be discovered with computational methods\cite{5,35} with a complete set of transcript boundaries.\cite{105,106,107} Solvent responsive regulatory motifs would be useful for designing a semi-synthetic stress response.\cite{45,46} If \textit{C. acetobutylicum} possesses a specific solvent-response system, an improved genome annotation could also enable the discovery of corresponding regulatory motifs.

The specific stress-response systems have unique detection and response mechanisms to particular intrinsic or extrinsic stressors. These genes have important applications in environmental remediation,\cite{128} pharmaceutical research,\cite{127} and renewables research. While a solvent stress response system has not yet been identified in solventogenic \textit{Clostridia}, a genome annotation could reveal stress-responsive genes and transcripts, beyond those identified by ORF predictions alone. In addition to these benefits, an improved genome annotation could also benefit knowledge of general stress response systems and will be discussed next. 


\subsection{General Stress Response Systems}

In contrast to specific stress-response systems, general responses are activated by more than one stimulus. For example, during both nutrient deprivation and acid stress, cells must slow or cease growth (stringent response) to adapt to energetic demands of the activation of both the specific and general stress response program. Also, both heat-shock and solvent stress can denature proteins and consequently activate chaperonin systems, another example of a general stress response.\cite{74,75,77} After detection of the stressor, signal transduction events activate dormant response machinery or activate/derepress response systems\cite{77,78}. These systems can be useful for stress response engineering\cite{45,46} and are somewhat conserved across genera, although their knowledge in \textit{C. acetobutylicum} remains incomplete. The general stress response is divided into four classes of genes based on the regulator responsible for their activation. 

\subsubsection{Class I}
The first class of general stress response genes is governed by the repressor HrcA, which responds to protein denaturation from thermal or chemical causes. The hrcA regulon contains at least 3 transcripts including the hrcA and dnaK/J, groES/EL, and htpG loci.\cite{42} Denatured proteins titrate the GroEL chaperone from HrcA/GroEL complexes, resulting in a conformational change of HrcA and decreased DNA binding.\cite{77,78} Operons regulated by the HrcA repressor are subsequently derepressed, rapidly increasing the amount of heat shock proteins. Protein denaturation negatively affects nearly every program and structure of the cell, resulting in decreased survival and viability. In \textit{C. acetobutylicum}, the HrcA motif was recently described for standard heat-shock operons.\cite{42} Additionally, class I genes are solvent-stress responsive due to solvent-induced protein denaturation.\cite{74,75} It is unknown if any additional operons are also regulated by this repressor in \textit{C. acetobutylicum}. To answer this question, an improved genome annotation including transcription start sites would facilitate the discovery of additional genes in the HrcA regulon through \textit{in silico} analyses.

\subsubsection{Class II}
The second class of genes is regulated by a stress-responsive $\sigma$-factor, $\sigma_{B}$. In \textit{B. subtilis}, the $\sigma_B$ regulon consists coordinates a general stress response for a variety of stressors. A \textit{C. acetobutylicum} $\sigma_{B}$ ortholog was not predicted in the initial genome annotation,\cite{91,42} although orthologous genes from its regulon have not similarly disappeared.\cite{132,133,134,135} The regulation of these genes is unknown in this organism. Perhaps $\sigma_{B}$ genes in \textit{C. acetobutylicum} are under the control of overlapping regulons, as in \textit{L. monocytogenes}\cite{131} or are regulated by an unknown mechanism. Given the large size of the $\sigma_{B}$ regulon and the presence of several of its genes (e.g. clpC\cite{132}), the promoter and regulatory regions of \textit{C. acetobutylicum} orthologs of $\sigma_{B}$-regulon genes would be useful for understanding the unique stress response of \textit{C. acetobutylicum}. This class of general stress-response genes present another opportunity for an improved genome annotation to facilitate stress response research.

\subsubsection{Class III}
The third group of genes are governed by the repressor CtsR, a dimeric helix-turn-helix regulator capable of responding to heat-shock, oxidative stress, and acid stress.\cite{136} In \textit{B. subtilis}, CtsR responds to heat-shock when ClpC, the fourth gene of the CtsR operon, releases McsB.\cite{140} Free McsB phosphorylates CtsR, leading to positive autoregulation and derpression of the ctsR regulon.\cite{137,138,140} This mechanism is thought to vary across the gram positive bacteria,\cite{137} but the operon organization suggests that the \textit{C. acetobutylicum} CtsR system is similar to \textit{B. subtilis}.\cite{42} A CtsR motif was found ahead of canonical CtsR regulon genes in \textit{C. acetobutylicum}(Qinghua paper) although additional genes may be controlled by CtsR in the absence of $\sigma_{B}$. A genome-wide motif search in this organism could similarly reveal CtsR regulation of solvent responsive transcripts. Similar to the class I system, knowledge of the ctsR regulon would also benefit from an improved genome annotation.

\subsubsection{Class IV}
The fourth and final class of general stress-response genes are regulated by unknown mechanisms.\cite{11} In \textit{C. acetobutylicuM}, this class includes stress-responsive genes with unconfirmed operon structure and no confirmed motifs. Microarray experiments from Venkataramanan \textit{et al.} show over 1,000 solvent responsive genes in \textit{C. acetobutylicum}, the regulation of which remains almost completely unknown.\cite{42} As suggested above, an improved genome annotation would aid the categorization of this massive gene set into class I, III, or potentially new regulons.

These general stress response programs are vital for adaptation and robustness. However, both sporulation and stress-response systems in the \textit{Clostridia} differ from the model for sporulating gram positive bacteria, \textit{B. subtilis}.\cite{126} At the very least, key genes are missing such as the sporulation kinases and the stress response regulator $\sigma$_B.\cite{126} Furthermore, most solvent responsive genes in \textit{C. acetobutylicum} are regulated by unknown motifs and response regulators.\cite{42} It is reasonable to expect that there may be some stress-response genes specific to the solventogenic \textit{Clostridia}. Such genes would have had little homology to known genes during the initial genome annotation and therefore would not be included in previous comparative genomic and microarray analyses.\cite{91,42} To illustrate the plausibility of this hypothesis, a recent study identified solvent responsive small RNAs in \textit{C. acetobutylicum}, many with no known homologs or regulators.\cite{39} Clearly, there is much to be done to understand and develop this organism for biofuel applications.

In this section, the example of stress-response systems was used to demonstrate differences of \textit{Clostridia} from the \textit{Bacillus} model. It is clear from this review of \textit{C. acetobutylicum} stress response systems, there are opportunities to discover novel transcripts or proteins not described by the initial genome annotation and provide precise transcript boundaries for motif identification. To provide definition to the stress response and other systems, regulatory regions and operon organization should be revealed by modern transcriptomic techniques. Techniques that provide these details throughout the genome are discussed in the following section.

\section{Transcriptomic Research}
Modern high-throughput techniques can produce gloabl datasets, even when reference genomes are not available.\cite{200} A large number of array and sequencing techniques have been developed to investigate the proteome, transcriptome, and interactome including the elements of protein-protein,\cite{141} protein-DNA,\cite{142,143,144} protein-RNA,\cite{145} and RNA-RNA interactions,\cite{146,147,149} protein post-translational modifications,\cite{149} single nucleotide polymorphisms,\cite{150} and transcript expression.\cite{151,152,153,154} These powerful genome-wide techniques allow experimental investigation of nearly every aspect of biological systems, including the identification of all genes and their boundaries. Here, we focus on transcriptomic techniques to identify these features. 

Transcriptomic techniques allow the characterization of the properties and dynamics of RNA species. Microarrays and sequencing techniques have revolutionized transcriptomic research, providing new insights into the complexity of the transcriptome\cite{152,154,155,156} and its regulation.\cite{157} In addition to the measurement of specific populations\cite{194,195}, specific features of the transcriptome can be quantified.\cite{193} Two common experimental approaches are used to explore cellular programs with high-throughput transcriptomic techniques: annotation and differential expression.

Before differential expression experiments are conducted, or in instances where a complete genome is not available, the catalog of all genes and their transcripts is required for microarray probe design or sequence read counting. ORF prediction algorithms\cite{158,159} and annotation suites\cite{160,161} can identify putative ORFs encoding enzymes and canonical genes for sequenced genomes, with small false positive and negative error rates. However, unique genes and small RNAs cannot be identified with these predictions. It is desirable to experimentally determine the transcriptome for conditions of interest.

The most common techniques for cataloging expressed transcripts and their properties are the tiling microarray\cite{162} and deep RNA sequencing\cite{115}. Microarray based analyses use probes spanning an entire genome, with some amount of overlap, to detect transcriptional activity.\cite{196} Deep RNA sequencing\cite{115} measures transcription through the number of cDNAs sequenced from fragmented RNA, roughly proportional to the expression level.\cite{197} Strand specific options for these methods are especially useful in dense bacterial genomes.\cite{162,115} The large amount of data require computational processing to identify the desired features.\cite{198,199} However, false positive and false negative errors (type I and II errors, respectively) from these techniques make automation of transcript annotation difficult, an issue infrequently discussed in the literature. Nevertheless, transcriptome assembly methods permit the reconstruction of full-length transcripts from sufficiently deep sequencing datasets at the expense of misassembly errors.\cite{108} In this section, transcriptomic techniques are reviewed and experimental designs discussed to identify opportunities for improvement over convention for this study's objective, revealing transcript boundaries, regulatory regions, and operon organization.

\subsection{Analytical Techniques}
\subsubsection{Microarrray}
Microarray-based transcriptomics allows the simultaneous measurement of thousands of sequences simultaneously.\cite{151} Probes designed for genomic sequences or open reading frames(ORFs) measure the expression of segments of the genome or ORFs, respectively. The former, tiling microarrays, are used to identify features of the transcriptome and annotate the genome. The latter is a typical practice for expression profiling experiments and identifying relative expression differences.\cite{151,162} The experienced research community have made the microarray an excellent platform for transcriptomic analyses.

Microarray technology has limitations that should be considered during the early stages of experimental design. First, the microarray suffers from limited detection range and sensitivity compared to RNA sequencing.\cite{163} Additionally, probe design is limited to either ORFs only or to overlapping segments of a genome sequence where available.\cite{153} Finally, the cost of tiling array experiments is a function of the amount of overlap between the probes; higher resolution implies higher cost.\cite{163} The limited detection range could make low-abundance transcripts challenging to identify or distinguish from spurious transcription.\cite{164,165}

A comprehensive investigation of \textit{B. subtilis} identified many ORFs to be expressed under various environmental and life-cycle conditions.\cite{162} The authors detected 85\% of all previously known transcription start sites.\cite{162} With 22-basepair resolution, tiling microarrays were used to detect increases in fluorescent signal across the genome, providing transcript boundaries after manual annotation. The use of manual methods to identify novel genes and transcript boundaries highlights the complexity of these datasets and the lack of methods for automated annotation. This study\cite{162} is an excellent model for transcriptome mapping using the microarray technology.

\subsubsection{RNA Sequencing}
The emerging technology of RNA sequencing(RNA-seq) is displacing tiling arrays for transcriptome mapping and annotation studies.\cite{163} Parallel sequencing platforms such as 454\cite{166} or Illumina\cite{167} enable the sequencing of gigabases of cDNA with precision for mapping and counting. cDNA libraries are sequenced in an extremely parallel manner, producing millions of sequenced ``reads'', which are then aligned to the genome. RNA-sequencing has a higher dynamic range than microarray technology in addition to basepair-level resolution.\cite{163} These characteristics make RNA-seq optimal for precise determination of transcript boundaries, even for low abundance transcripts.

To achieve this, each step of the experiment and RNA processing must be considered carefully. Poor consideration of the factors involved with this method leads to uneven coverage,\cite{108,109,110,111,115} poor depth,\cite{114,115} and questionable strand specificity.\cite{113} While attempts have been made to establish standards for biomedical RNA-seq,\cite{110} no guidelines exist for the broader research community. RNA-seq is frequently performed with the Illumina platform for the amount of data it produces. This amount of data plays a crucial role in detecting transcript boundaries and low abundance transcripts.

A central challenge to the design of RNA-sequencing experiments is related to detection limit. The likelihood of sampling low-abundance cDNA fragments from transcript termini or low abundance transcripts is a function of the number of sequenced reads.\cite{108,109,110,112} The absence of reads at a location in the genome indicates either an insufficient sequencing depth(false negative or type II error) or a true absence of transcription. Bacterial cDNA libraries are commonly sequenced in multiplex over one or more lanes of an Illumina sequencer, leading to millions of reads distributed across the libraries. Therefore, there is a three-part tradeoff between the replication, depth, and independent variables in the experimental design. A recent study concerning the tradeoff between the first two concluded that for differential expression experiments, replication is preferable.\cite{112} In the case of transcriptome mapping and genome annotation however, additional depth may be preferable.\cite{110}

Standard procedures are required to compare sequencing depth between RNA sequencing studies with similar objectives (e.g. transcription start sites). The Encyclopedia of DNA Elements(ENCODE) research consortium frequently uses RNA-seq in various forms\cite{168,169} and has published best practice guidelines for Human genome research.\cite{110} While average per-base sequencing depth and their distributions are not provided, they suggest that 100-200 million clusters of paired-end reads is sufficient depth to identify novel transcripts and transcriptomic features (e.g. TSSes) in the hg19 \textit{H. sapiens} transcriptome. This number is much larger than required for microbial genomes; a preferable metric is the ratio of the number of reliably-mapped non-ribosomal reads to the approximate size of the transcriptome. Next, this metric is used for persepective to compare transcriptome mapping studies.

\subsection{Transcriptome Mapping Studies and Common Challenges}

\begin{table}
\begin{center}

\resizebox{\textwidth}{!}{%
\begin{tabular}{|l|c|c|c|c|c|c|c|}\hline
 & Genome(Mbp) & Transcriptome(Mbp) & Clusters(M) & rRNA-free(M) & Mapped(M) & Alignment Rate & $Ratio^{\dagger}$\\\hline\hline
Standard\cite{110} & 3000 & 140 & 30 & 30 & 25 & N/A & 0.18\\\hline
Deep\cite{110} & 3000 & 140 & 100 & 100 & 100 & N/A & 0.71\\\hline
Ultra-deep\cite{110} & 3000 & 140 & 200 & 200 & 200 & N/A & 1.43\\\hline
\textit{C. beijerinckii}\cite{114} & 6 & 6-12 & 14 & N/A & 11.5 & 0.82 & N/A\\\hline
\textit{P. difficile}\cite{170} & 4.3 & 4-8 & 50 & 4 & N/A & N/A & 0.5-1\\\hline
\textit{B. anthracis}\cite{171} & 5.5 & 5-11 & 33 & N/A & 5 & 0.15 & N/A\\\hline
\textit{H. pylori}\cite{113} & 1.7 & 1.5-3 & 0.4 & N/A & 0.2 & 0.54 & N/A\\\hline
\textit{Synechocystis. sp}\cite{119} & 3.9 & 3.9-7.8 & 0.2 & 0.1 & 0.1 & 0.475 & 0.03\\\hline
\textit{E. coli}\cite{115} & 4.6 & 4.5-9 & 52 & N/A & 17.7 & 0.34 & N/A\\\hline
\textit{P. gingivalis}\cite{172} & 2.3 & 2.3-4.5 & 15 & N/A & 2.3 & 0.15 & N/A\\\hline
\textit{S. typhi}\cite{173} & 5 & 5-10 & 5.7 & N/A & 1.8 & 0.31 & N/A\\\hline
\end{tabular}
}
\end{center}
\caption{Study Comparison: Poor Data Utilization Rates}\label{table:study_compare}
\small
RNA-seq transcriptome mapping studies have widely different sequencing depths and alignment rates. These depths are not directly comparable between organisms; rather, the number of clusters/reads divided by the size of the transcriptome$^{\dagger}$ is a more ideal metric for comparison. However, there is little discussion of the diluting effect of rRNA on useful sequencing depth, such as the 8\% utilization in a study in \textit{P. difficile}.\cite{115} As a result, the ratio$^{\dagger}$ cannot be calculated precisely for all studies. That being said, the poor read alignment and rRNA removal rates suggest that many studies do not achieve desirable sequencing depth given the size of the transcriptome.
\end{table}


The most phylogenetically similar organism to \textit{C. acetobutylicum} that has been investigated with RNA-seq transcriptome mapping is \textit{C. beijerinckii}.\cite{114} In this study, 82\% of the reads aligned uniquely to the genome, a good alignment rate and depth when compared to other studies (Table \ref{table:study_compare}). However, the authors fail to qualify their work with respect to several factors that influence transcript discovery and transcription start site identification. These common issues to transcriptome mapping studies are described next.

In dense bacterial genomes, proteins are coded by polycistronic transcripts, operons, that are packed closely in to small circular genomes, only a few megabases in length. In bacteria, overlapping transcripts can be encoded on opposite strands. Strand specific techniques are seldom used,\cite{115} preventing the detection of divergent operons, novel genes in antisense, and \textit{cis}-encoded sRNAs.

Unlike eukaryotic transcriptome mapping studies where poly-A selection is available, bacterial total RNA extracts contains 95-99\% ribosomal RNA.\cite{116,117,118} For most RNA-seq applications, the overwhelming presence of rRNA lowers the useful sequencing depth extraordinarily. Several commercial kits are available with suboptimal and inconsistent removal rates.\cite{115,116,117} Very few studies provided calculations or discussion of the effect that rRNA had on the effective sequencing depth.

Another intrinsic artifact to the RNA sequencing method is the effect of preferential PCR amplification on library complexity.\cite{174,175} After cDNA library construction, the library is typically amplified to provide additional material for sequencing. Duplicated reads provide redundant information and should be removed \textit{in silico}.\cite{40} It seems that no efforts were made to address this issue in these studies.  

Ribosomal RNA and PCR-amplification bias are two sources of noise that lower the amount of useful signal from bacterial RNA-seq experiments. \textit{In silico} solutions\cite{40,17} allow researchers to separate and quantify the useful signal for the purposes of qualification and comparison with other studies. Ribosomal RNA was treated in two of eight studies reviewed, showing poor data utilization.\cite{119,170} Correction for duplicate reads from amplification bias was not found. Therefore, the true quantity of useful data in many studies is unknown. Furthermore, it is unclear what the distributions of per-base sequencing depth actually were in these studies. Quantification of the sensitivity used by these studies(Table \ref{table:study_compare}) would have helped the experimental design for this work.

The final issue concerns the presence of background signal. Transcriptional noise is a phenomenon that is detectable with high sensitivity methods such as deep RNA sequencing.\cite{164,165} RNA sequencing can detect low abundance signals such as residual genomic DNA,\cite{176} low abundance transcripts,\cite{109,110,58,177} and spurious transcription.\cite{164,165} It is unclear how this noise was distinguished from signal in these studies, despite low and unstable depth of coverage reported by many studies.\cite{113,114,115,172} In the most sensitive study reviewed (Table \ref{table:study_compare}) the authors report ``...less than 60\% genes in the genome had their length completely covered by at least one read''.\cite{115} The effect of background signal on false positive or type I error is largely ignored in the literature.

These issues highlight the need for standards in RNA sequencing and transcriptome mapping, similar to MIAME standards for microarrays.\cite{178} Clearly there are a number of challenges for transcriptome mapping studies, especially with unknown type I and type II error rates typical of exploratory projects.\cite{108,109,110,111,112,174,175,176,177,179} However, by addressing technical obstacles explicitly with bioinformatic methods, the amount of useful data can be quantified. Taken together, these issues informed the experimental design and qualification of sensitivity for this project.

\section{Lessons and Objectives}
The biofuel producing bacterium \textit{C. acetobutylicum} is an excellent platform organism for bioprocesses. A crucial challenge for productivity in this and other organisms is the tolerance of the host to large biofuel concentrations. Knowledge of the stress response systems of this organism are limited by an antiquated genome annotation\cite{91} without transcription start sites, operon organization, or promoter signals. These transcriptomic features are essential for understanding coexpression and regulatory networks. RNA sequencing is a powerful method used here to identify these features for future research on stress response systems and biofuel tolerance. This project aims to identify transcription start sites in the \textit{C. acetobutylicum} genome while addressing frequently ignored issues in sequencing approaches. 

