% This is the file for the background chapter of my thesis

\chapter{Background}

\section{History of the Involvement of Clostridium acetobutylicum in Industrial Processes}
Industrialization and suburbanization have led to dramatic increases in CO2 levels over the 20th century.(REVIEWS) Current climate models suggest that increases in ocean levels, temperature, glacial melt/polar ice reserves, variations in weather patterns are attributable to the rise in CO2 levels.(REVIEWS) Meanwhile, petrochemical exploration suggests a dwindling amount of new prospects. (CITATIONS) There is both an economic and moral imperative to developing renewable strategies for fuel and chemical production.

Our modern and behemoth petrochemical system is limited by finite petroleum reserves. The renewables industry requires processes that utilize abundant and economical resources. Good feedstocks have CO2 as the ultimate Carbon source in the supply chain. The best microorganisms for these conversions can consume a diverse range of substrates and convert to product at high yield. An excellent biofuel would be renewable, economical, and compatible with existing combustion technology and infrastructure. Government agencies (EPA,DOE) have encouraged the growth of the biofuels industry with grants, tax credits, and other financial incentives. The history of recent biofuels development is briefly explored.

First generation biofuels such as ethanol and conventional biodiesel use sugars and fats and input, thus affecting the supply of food. Ethanol has additional issues concerning its hygroscopicity and energy content. It decreases combustion efficiency (source?) and has a negative effect on many conventional systems when used in high concentrations. There have been additional concerns raised about the net energy benefits of ethanol (PNAS study, others, DOE?). The first generation of biofuels compete with consumers for supply and fail to use more abundant sources of carbon.
Second generation biofuels focus on the use of non-food crops and waste biomass. Many fuel molecules have been considered, but lignocellulosic ethanol is the classic example. Producing fuels from non-food crops and industrial byproducts has the advantage of cheap feedstock. However, the processes required to digest lignin and cellulosic carbon sources tend to involve expensive pretreatments (examples? sources?). So far, US companies have failed to meet EPA/DOE second generation fuel production requirements by X gallons. An inexpensive process for lignocellulosic digestion coupled to biofuel conversion would have clear advantages over current pretreatments and the ethanol industry.

C acetobutylicum can consume lignocellulose (source), a wide variety of simple and complex carbohydrates, and can synthesize most amino acids from a suitable nitrogen source. It has long been valued as an industrial solvent producer and has a set of advance genetic tools for metabolic and biosystems engineering. Moreover, it is a native producer of butanol, an advanced biofuel. Butanol and its analogues hav been identified as prime targets for biofuels research, currently investigated by DuPont, BP, Gevo, and others. This microbe meets the requirements for low-cost non-food feedstocks and infrastructure compatibility. Therefore, C. acetobutylicum is an excellent chassis for an integrated biorefinery.

Clostridium acetobutylicum is a species historically used for solvent production in the 20th century.(REVIEWS) It belongs to a genus of anaerobic soil bacteria of industrial and pathogenic importance. It is known as the model for the Acetone-Butanol-Ethanol(ABE) fermentation, notably used in the Weizmann process of the first World War (REVIEWS). The Weizmann process produced 6 parts of butanol, 3 parts acetone, and 1 part ethanol in an anaerobic fermentation of starch, molasses, and other substrates. Its feedstock flexibility made it an excellent microbe for production during the first and second World Wars. Prized for its unique fermentation and tolerance of a variety of inputs, this organism remained the top producer of short-chain alcohols and solvents for decades, until the development of improved petrochemical processes.

Current research efforts are reviving this fermentation process with new genomic tools and analyses. The rate of genome sequencing efforts (NCBI stats) has resulted in (32?) sequenced Clostridia, providing insight into the metabolism and systems of the interesting species in this genus. Modern high-throughput sequencing has provided information about C. acetobutylicum's relatives and genetic tools (review) have enabled metabolic and biosystems engineering efforts in this genus. Metabolic engineering efforts have been made to increase the carbon flux towards the primary products.(Examples?) Efforts have been made to increase the carbon yield towards 1.(S.Y. Lee?) Attempts have been made to optimize the lignocellulosic machinery (source?) or to introduce alternative metabolic machinery (glycerol, wood ljungdahl).

Some attention has also been devoted towards increasing productivity of ABE fermentations by phenotype engineering. Acetone, butanol, ethanol, and other microbial waste products are produced during fermentation. These hydrophobic and amphipathic metabolic products intercalate cellular membranes and facilitate protein misfolding. This disrupts the electrochemical gradients of these membranes, leading to increased energy demands and changes to membrane stability. Also, increased protein misfolding affects nearly every process of the cell. Accumulation of these products induces a stress response to counteract the harmful effects of these metabolites. (sources) The systems responsible for solvent tolerance are complex and many specific responses remain poorly understood or unknown. The understanding of these systems is the goal of this study and I will briefly review the stress response in bacteria, focusing on its mechanisms and responses.

\section{Bacterial Stress Response and Solvent Tolerance}

Bacteria respond to a wide variety of environmental challenges through stress response systems. These networks of genes respond to signals indicating a hostile or desolate environment. Both gram negative and gram positive bacteria have developed physiological and genetic adaptations to these stressors. These adaptations generally slow growth rate, conserve resources, and prolong the viable life of the cell. Conditions of plenty in the natural environment are rare (source), and cells have developed adaptations to specific stimuli of the natural environment. Thus these adaptations are capable of detecting specific stressful conditions and transforming the signal into the activation of stress response systems, network of genes that enable cells to adapt and survive. Many stressful conditions exist, yet they all induce both specific and general stress responses.

Specific stress responses begin with the detection of a stressful environment. The absence of certain amino acids, minerals, vitamins, or energy may impair the cells ability to grow or survive. Cells can detect energy levels through intracellular messengers (and voltage gated ion channels?), activating programs that conserve resources and slow growth. Dangerous environmental conditions such as temperature fluctuations, oxidative and osmotic stress trigger other response systems that protect the cells from protein, DNA, or chemiosmotic gradient damage. Finally, cells also encounter biological, inorganic, and organic toxins that activate stress response programs to encourage proper protein folding, toxin export, and more. 

There is one thing that most of these programs have in common: the general stress response. Many of these specific response programs also induce a general stress response. These programs provide non-specific survival benefits to others stressors and constitute many protective and conservational responses to environmental stress. Common objectives of stress response are met through the activation of general stress response programs. For example, during both nutrient deprivation and acid stress, cells must slow or cease growth to cope with changing energetic demands and the metabolic burden of the activation of both the specific and general stress response program. Proceeding with this example, I highlight one of the general stress response programs.

The stringent response may begin with a decline in the availability of an amino-acid or mineral (e.g. phosphorous, bioavailable nitrogen) in the environment. Decreased levels of the amino acid are observed in a cell, typically by riboregulators/riboswitches. These activate regulatory

Next, ppGpp accumulation modulates the expression of a large number of genes involved in primary metabolism, growth, replication, and competence (survival). This regulation is mediated in part by the RNA-pol holoenzyme itself and with some help from dksA (what is the role of dksA?). The stringent response was also shown to crosstalk with the Csr carbon-storage pathway to modulate the metabolic rate.




The general or non-specific stress response is induced after a specific stress response has begun. After detecting the stress, signal transduction events activate the general stress response system. Currently, the general stress response is divided into four classes of genes. The first class of genes is governed by the repressor HrcA. When the cell is stressed by one or more types of compatible stresses, the HrcA regulon is derepressed, increasing the expression of proteins such as the dnaK and groEL chaperonins. The second class of genes is 



The general or non-specific stress response is induced by a variety of stimuli, such as energy starvation, or mineral, nutrient, or amino-acid deprivation. Specific signals activate two component signaling systems in the cell, which then relay the signal to response regulators or transcription factors. Specific and non-specific programs are thus activated, increasing the tolerance of the cell to its environment. An interesting finding (source) in the (1990s??) was that specific stresses produced non-specific survival benefits to other types of stresses. These non-specific tolerance systems constitute the general stress response. A typical example of this response include the so-called stringent response, which inhibits replication, reduces metabolic rate, and conserves energy.

A second system that is induced by multiple conditions and produces survival benefits to more than one type of stress is the heat-shock response. 

The heat-shock response is most induced by conditions of high temperatures (> 40 degrees) but may also be induced by osmotic, starvation, or other conditions (sources). This conserved system includes both specific and non-specific members (1990 general stress response of bsub). Canonical examples in this sample include the GroEL chaperonin machine (source) and proteases. The GroEL chaperonin system is a highly conserved group of proteins, forming a multimer. Through cycles of binding, ATP hydrolysis, and structural changes, the chaperonin proteins guide malformed proteins to their unfolded state and back again to encourage proper folding. Protein misfolding is increased not only under heat stress, but in solvent and pH stress as well. The heat-shock system is therefore particularly useful for biotechnological applications.




GENERAL STRESS RESPONSE
A recent experiment showed an increase in final fermentation yield (productivity) by overexpressing certain stress response proteins and chaperones. These efforts follow engineering efforts in other organisms (e.coli source??). The increased productivity of these strains suggest that an abundance of chaperonins, proteases, and other stress response proteins may counteract the incrased protein misfolding rate due to abundant solvents and acids.


SPECIFIC STRESS RESPONSE


\section{History of Genomic and Transcriptomic Research}
