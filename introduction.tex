%
% This is Chapter 1 file (chap1.tex)
%
\chapter{Introduction}


\section{Motivation}
Increases in global CO2 levels, sea level, temperature, and acidification are tipping climate models toward disaster.(UN AR5). Few solutions exist for what has been described as the "...issue that will define the contours of this country more dramatically than any other." (OBAMA) A chief issue with the global CO2 equation is the lack of systems which consume the greenhouse gas. A renewable chemicals industry has been suggested to restore balance to our climate system. No economically effective system exists to convert CO2 into high-quality fuels and chemicals, though the magnitude of the first-generation biofuels industry has demonstrated the scalability of bioconversions.

Leading scientists and engineers (Liao, S.y.Lee, Papoutsakis) recognize Clostridum acetobutylicum as a potential platform organism for a 'biorefinery'. This organism produces an advanced biofuel called butanol, a gasoline replacement being investigated by Gevo, DuPont, and BP. Over the last (X) years, genetic tools have been developed to optimize C. acetobutylicum productivity. Metabolic engineering techniques are being used to increase butanol yield and its already impressive feedstock flexibility. However, successful strains also require biosystems engineering to increase robustness and stress tolerance. Limited knowledge of the stress response systems in C. acetobutylicum is a barrier to engineering robust strains for renewable fuel production.

\section{Approach}
To address this issue, I investigated the response of the transcriptome to biofuel and metabolite stress using Next-Generation Sequencing. I used a mixture of laboratory and informatic approaches to understand the restructuring of the transcriptome under small-molecule stress. RNA-sequencing allowed both the assembly and categorization of transcripts and small RNAs. To make these data accessible to metabolic and biosystems engineers, I constructed interactive visualizations for these results including MA-plots, significance criteria, clustering results, and coverage vectors.

\section{Document Overview}
This document describes the laboratory and informatic procedures used to generate a comprehensive view of a state of the transcriptome and how to compare these states to generate hypotheses. Here I describe the first reported transcriptome assembly and annotation in Clostridium acetobutylicum, a model gram-positive anaerobe and platform organism for biofuel production. Additional results include novel transcripts, CDSes (possible new systems), and TFBSes. I discuss hypotheses that could explain the interesting observations. I discuss the impact of this work and how it should be used to increase productivity in microbial fermentations.