%
% This is Chapter 1 file (chap1.tex)
%
\chapter{Introduction}


\section{Motivation}
Increases in global CO2 levels, sea level, temperature, and acidification are tipping climate models toward disaster.(UN AR5). Few solutions exist for what has been described as the "...issue that will define the contours of this country more dramatically than any other." (OBAMA) A chief issue with the global CO2 equation is the lack of systems that utilize this greenhouse gas. A renewable chemicals industry has been suggested to restore balance to our climate system in an economically sustainable way.

Leading scientists and engineers (Liao, S.y.Lee, Papoutsakis) recognize \textit{Clostridum acetobutylicum} as a potential platform organism for a biorefinery, a bioprocess featuring an organism which converts different substrates into chemicals. This microbe produces an advanced biofuel called butanol, an energy-dense gasoline replacement. Recently, genetic tools have been developed to optimize \textit{C. acetobutylicum} productivity(source). Metabolic engineering techniques are being investigated(source) to increase butanol yield and its already impressive feedstock flexibility. Successful strains also require biosystems engineering to increase robustness and fuel tolerance. Limited knowledge of the biofuel-stress tolerance systems in \textit{C. acetobutylicum} is a barrier to engineering robust strains for renewable fuel production.

Research and development in \textit{C. acetobutylicum} requires a complete and experimentally determined genome annotation. A complete set of transcript boundaries facilitates research into biochemical and molecular systems such as biofuel tolerance. Transcription start sites(TSS) enable the discovery of regulatory motifs and network structure. Only open reading frames were predicted in the original genome annotation, which has not been verified by absolute expression measurements. To provide the features of interest, a transcriptome mapping study was designed.

Transcriptome mapping studies frequently utilize high-depth Next-Generation Sequencing methods to provide sensitivity for low abundance transcripts and their features. However, the required depth of sequencing for high-sensitivity bacterial transcriptome mapping is unknown and authors avoid its discussion(citations), instead mistaking biased fold-coverage estimates for sequencing depth. Moreover, studies often fail to quantify ribosomal RNA removal rates(rRNA removal citations) which have a dramatic effect on effective sequencing depth. Additionally, these studies rely on sequencing depth alone, ignoring the complexity of the dataset for assembly(cite transcriptome assembly rev). Without the empirical guidelines described for Eukaryotic applications(Cite encode), these studies under(cite 454 studies) or over-estimate(E. coli, C beij) the required amount of sequencing yet observe low complexity(E.coli study) and poor utilization(ref table comparison) of their sequencing results.


\section{Approach}
To facilitate future renewables research in \textit{C. acetobutylicum}, an RNA sequencing(RNA-seq) transcriptome mapping study was designed with sensitivity in mind. A mixture of laboratory and informatic approaches were used to identify and remove ``noise''(e.g. rRNA, duplicate reads) from the sequencing library, leading to a high-quality sequencing dataset and unbiased sequencing-depth statistics. Transcriptome assembly was used to describe operon structure and estimate transcript boundaries. Finally, a genome browser was constructed to identify and resolve misassemblies and share the data with the \textit{Clostridia} research community.


\section{Document Overview}
This thesis describes a technique for unbiased and precise estimation of transcript boundaries in microbial systems. The document compares related approaches for transcriptome mapping and identifies their strengths and weaknesses. A methodology is described that lead to a high-quality sequencing dataset, optimized with considerations of biological and technical noise. After describing this technique, the sensitivity of the technique is qualified both in terms of fold-coverage and sequencing depth. Next, an assessment of the assembled dataset is presented, which reveals challenges associated with high-depth sequencing, either not detected or described by comparable studies. This assessment lead to the development of a genome browser that was used for proof-of-principle curation of the pSol1 megaplasmid, markedly improving the type I error rate. This document describes a method and its application in the model solventogenic bacterium \textit{C. acetobutylicum}, marking the first reported strand-specific transcriptome assembly in the \textit{Clostridia}. This thesis describes the discovery of novel genomic and transcriptomic features that will be shared through the genome browser for the entire \textit{Clostridia} and renewables research community.

