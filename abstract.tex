% Abstract.tex


The {\it C. acetobutylicum} genome annotation has been markedly improved by integrating bioinformatic predictions with RNA sequencing(RNA-seq) data. Samples were acquired under butanol, butyrate, and unstressed treatments across various growth stages to sample the transcriptome from a range of physiologically relevant conditions. Analysis of an initial assembly revealed errors due to technical and biological background signals, challenges with few solutions. Hurdles for RNA-seq transcriptome mapping research include optimizing library complexity and sequencing depth, yet most studies in bacteria report low depth and ignore the effect of ribosomal RNA abundance and other sources on the effective sequencing depth. 

In this work, workflows were established to address type I and II errors associated with these challenges. An integrative analysis method was developed to combine motif predictions, single-nucleotide resolution sequencing depth, and library complexity to resolve these errors during assembly curation. This contextualization minimized false positive error and determined gene boundaries, in some cases, to the exact basepair of prior studies. Curation of the pSOL1 megaplasmid reconciled transcriptome assembly statistics with findings from \textit{E. coli}. 

The resulting annotation can be readily explored and downloaded through a customized genome browser, enabling future genomic and transcriptomic research in this organism. This work demonstrates the first strand-specific transcriptome assembly in a \textit{Clostridium} organism. This method can improve the precision of transcript boundary estimates in bacterial transcriptome mapping studies. 