% 
% This is the Title and Approval Page file (thesis-tap.tex) for
% a master's thesis.
%
% The order of the commands below is very important.
% You may choose to add or eliminate a \prefacesection 
% in the front material but the order should remain 
% the same especially \maketocloflot followed by 
% \prefacesectiontoc{Abstract}

% Title and author are also used for PDF file properties
% No special character or commands can be used for the PDF definition; 
% use the [options] paramater to specify a different title or author 
% to remove special characters or commands like \\ for example.
\title[Assembling Improved Gene Annotations in Clostridium acetobutylicum with RNA Sequencing]{Assembling Improved Gene Annotations in \textit{Clostridium acetobutylicum} with RNA Sequencing}
\author{Matthew T. Ralston}
\type{thesis}
\degree{Master of Science}
\majorfieldtrue\majorfield{Bioinformatics and Computational Biology}
\degreedate{Winter 2015}
% Optional PDF properties
\keywords{Transcriptome,RNA-seq,Assembly,Annotation,Genomics}
\subject{Master of Science in Bioinformatics and Computational Biology}

\maketitlepage % Generates Title Page

\begin{approvalpage}
\prof{Eleftherios T. Papoutsakis, Ph.D.}
\chair{Errol Lloyd, Ph.D.}{Chair of the Department of Computer Science}
\dean{Babatunde Ogunnaike, Ph.D.}{Dean of the College of Engineering}
\end{approvalpage}

\begin{front} % Starts front material (Roman style page numbers)

\prefacesection{Acknowledgments}

%					A C K N O W L E D G E M E N T S
% Acknowl.tex

I write this paper with unending thanks for my family, friends, the love, support and encouragement that they give, and the lessons they have taught me. With special thanks for my mother Donna, father Thomas, and sister Allison. With special thanks to my mother; her character and duty towards others is inspirational for this work's focus on sustainability and climate change. With special thanks to my Father, for inspiration through his work ethic, leadership, and open mind. With special thanks to my sister for her support of my education, curiosity, and maturity; I would not be where I am without her encouragement and acceptance. I write this with gratitude to my family for the life they have given me, the sacrifices they have made on my behalf, and most importantly their love. With special thanks for my grandfather George inspiring my pursuit of science and chemistry. With special thanks for my grandmother Winnie for the inspiration of her love and optimism which bring me through each challenge. With special thanks for my uncle Kevin for his curiosity, friendship, optimism, and support. With special thanks for my brother-in-law Matt for his friendship and encouragement. With special thanks for my nieces Violet and Ruby for their love, for the lessons that they teach me, and their infectious energy and optimism. Many thanks for Madeline for her love, open ear, curiosity, support, and encouragement throughout the years.  Many thanks to my best friend Andrew for his friendship, curiosity, support, and encouragement. With many thanks to the rest of my wonderful and supportive family and friends for their unconditional love and support. 

Many thanks to Karol Miaskiewicz for his consistent and wonderful friendship throughout this project. Many thanks to all of the members of the Bioinformatics program. Particularly, I'd like to thank Erin Crowgey for her mentorship and support of my efforts to learn NGS bioinformatic analyses and Ryan Moore for tossing ideas around together. I'd like to thank Shawn Polson for his open ear and perspective. I'd also like to thank Dr. Wu for her support and encouragement throughout the program and the many members of the Wu group for their support as well. Many thanks to Bruce, Olga, and Summer for their work in the Sequencing and Genotyping Center in support of this project.

Many thanks for my mentors, Drs. Keerthi Venkataramanan and Terry Papoutsakis, for their mentorship, support, critique, and understanding throughout this project. The success that we've seen in this effort I owe to Keerthi's guidance and training. 
Many thanks for each of the the many members of the Papoutsakis lab for their friendship, support, and critique. I am very grateful for the unity of this group and for how much I enjoy coming to the lab each day. For the hard work, sleepless nights, early autoclave cycles, shared spaces, and plenty of reasons to celebrate I am grateful for each member.

{\centerline {\it Ad maiorem dei gloriam.}} % This file (acknowl.tex) contains the text
                % for the acknowledgments.


% Table of Contents is always created, but you
% may set \tablespagefalse and \figurespagefalse 
% if you don't want these generated automatically
% (i.e. List of Tables and List of Figures).
% These are set to true by default (i.e. \tablespagetrue,
% \figurespagetrue).

% Uncomment if you do not want a List of Figures.
%\figurespagefalse

% Uncomment if you do not want a List of Tables.
%\tablespagefalse 

\maketocloflot

\prefacesectiontoc{Abstract}

%					A B S T A C T
%% Abstract.tex


The {\it C. acetobutylicum} genome annotation has been markedly improved by integrating bioinformatic predictions with RNA sequencing(RNA-seq) data. Analysis of an initial assembly revealed errors due to technical and biological background signals, challenges with few solutions in the genomic literature. Additional hurdles for next-generation sequencing(NGS) based transcriptome mapping research include optimizing library complexity and sequencing depth, yet most studies in bacteria report low depth and ignore the effect of ribosomal RNA abundance and other sources on the effective sequencing depth. 

In this work, \textit{in vitro} and \textit{in silico} workflows were established to address false positive and negative errors associated with transcriptome mapping. An integrative analysis method was developed to integrate motif predictions, single-nucleotide resolution sequencing depth, and complexity during curation. This contextualization minimized false positive error, providing the precise and accurate determination of gene boundaries qualitifed by previous studies, in some cases, to the exact basepair. Curation of the pSOL1 megaplasmid improved statistical measures of transcriptomic features to be amenable with findings from \textit{E. coli}. 

The resulting annotation can be readily explored and downloaded through a customized genome browser, enabling future genomic and transcriptomic research in this organism. This work demonstrates the first strand-specific transcriptome assembly in the \textit{Clostridia}. Additionally, this method can be used to eliminate false positive features from assemblies in bacterial transcriptome mapping studies.  % This file (abstract.tex) contains the text
                 % for an abstract.
% Abstract.tex


The {\it C. acetobutylicum} genome annotation has been markedly improved by integrating bioinformatic predictions with RNA sequencing(RNA-seq) data. Analysis of an initial assembly revealed errors due to technical and biological background signals, challenges with few solutions in the genomic literature. Additional hurdles for next-generation sequencing(NGS) based transcriptome mapping research include optimizing library complexity and sequencing depth, yet most studies in bacteria report low depth and ignore the effect of ribosomal RNA abundance and other sources on the effective sequencing depth. 

In this work, \textit{in vitro} and \textit{in silico} workflows were established to address false positive and negative errors associated with transcriptome mapping. An integrative analysis method was developed to integrate motif predictions, single-nucleotide resolution sequencing depth, and complexity during curation. This contextualization minimized false positive error, providing the precise and accurate determination of gene boundaries qualitifed by previous studies, in some cases, to the exact basepair. Curation of the pSOL1 megaplasmid improved statistical measures of transcriptomic features to be amenable with findings from \textit{E. coli}. 

The resulting annotation can be readily explored and downloaded through a customized genome browser, enabling future genomic and transcriptomic research in this organism. This work demonstrates the first strand-specific transcriptome assembly in the \textit{Clostridia}. Additionally, this method can be used to eliminate false positive features from assemblies in bacterial transcriptome mapping studies. 

\end{front}

